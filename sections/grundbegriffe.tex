\section{Grundbegriffe}
	\subsection{Einfaches Kommunikationsmodell}
		\begin{center}
			\begin{tikzpicture}[scale=1, transform shape]
    \node [sum]   (source) {Quelle Source};
    \node [block, right of=source]  (sender) {Sender Transmitter};
    \node [block, right of=sender]  (channel) {Kanal Channel}; 
    \node [block, right of=channel] (receiver) {Empfänger Receiver}; 
    \node [sum, right of=receiver]  (sink) {Senke Sink};
    
    \node [block, above=5mm of sender.north east] (noise) {Störung Noise};    
    \draw [pfeil, to path={-| (\tikztotarget)}] 
    	(noise) to (channel);
    	
    \draw [pfeil] (source) -- 
    	node[above=1mm] {Nachricht} 
    	node[below=1mm] {Message}
    	(sender);
    \draw [pfeil] (sender) -- node[above=1mm] {Signal} (channel);
    \draw [pfeil] (channel) -- node[above=1mm] {Signal} (receiver);
    \draw [pfeil] (receiver) -- 
    	node[above=1mm] {Nachricht} 
    	node[below=1mm] {Message}
    	(sink);
\end{tikzpicture}
		\end{center}
\subsection{Signalklassifizierungen}


\renewcommand{\arraystretch}{2}
\begin{tabular}[c]{ | p{9cm} | p{9cm} | }
\hline
	\begin{minipage}[t]{9cm}
		\textbf{Energie} \\
		$ E = W = \int\limits_{-\infty}^{\infty} |x(t)|^2dt\label{SIG_FORM_01}
		 = \frac{1}{2 \pi} \int\limits_{-\infty}^{\infty} |X(j \omega)|^2 d \omega$ \\
	\end{minipage}
	&
	\begin{minipage}[t]{9cm}
		\textbf{Leistung} \\
		$ P = \lim \limits_{T \to \infty} {\frac{1}{T} \int\limits_{-T/2}^{T/2} {|x(t)|^2 dt}} 
		= \frac{1}{2 \pi} \int\limits_{-\infty}^{\infty} \left( \lim\limits_{T
	\rightarrow \infty} \frac{|X(j \omega)|^2}{T} \right) d \omega \\= \sum_{n=-\infty}^{\infty} |c_n|^2	$  (Parsevalsches Theorem) \\
	\end{minipage}
	\\
\hline

	\begin{minipage}[t]{9cm}
		\textbf{Energiesignal} - \textit{''Impuls'' bspw. Nachrichtensignal}
		\begin{center}
			$ E < \infty $ \\
			\includegraphics[width=4cm]{bilder/signal_energiesignal.png}
       	\end{center}

	\end{minipage}
	&
	\begin{minipage}[t]{9cm}
		\textbf{Leistungssignal} - \textit{''Dauersignal '' bspw. Trägersignal}
		\begin{center}
			$ E = \infty \text{ und } P < \infty$\\
			\includegraphics[width=4cm]{bilder/signal_leistungssignal.png}
       	\end{center}

	\end{minipage} \\

\hline

	\begin{minipage}[t]{9cm}
		\textbf{Aperiodisch} 
		$x(t) \neq x(t + n \cdot T)$
	\end{minipage}
	&
	\begin{minipage}[t]{9cm}
		\textbf{Periodisch} 
		$x(t) = x(t + n \cdot T) \qquad \text{ Periodendauer } T \text {; } n \in \mathbb{Z}$
	\end{minipage}
\\
\hline

	\begin{minipage}[t]{9cm}
		\textbf{Deterministisch} - \textit{mit vorbestimmten Verlauf} 
		$x(t) = f(t)$
	\end{minipage}
	&
	\begin{minipage}[t]{9cm}
		\textbf{Stochastisch} - \textit{ohne vorbestimmten Verlauf} 
		$x(t) = ?$
	\end{minipage}
\\
\hline

	\begin{minipage}[t]{9cm}
		\textbf{Zeitkontinuierlich} \\
		$x(t) \text{ ist definiert } \forall t \in \mathbb{R}$
		\begin{center}
			\includegraphics[width=4cm]{bilder/signal_zeitkontinuierlich.png}
       	\end{center}
	\end{minipage}
	&
	\begin{minipage}[t]{9cm}
		\textbf{Zeitdiskret} \\
		$x(t) \text{ nur definiert an Stellen } x(n \cdot T) $ \\
		$  \text{ mit Abtastintervall } T \text { und } n \in \mathbb{Z}$
		\begin{center}
			\includegraphics[width=4cm]{bilder/signal_zeitdiskret.png}
       	\end{center}
	\end{minipage}
\\
\hline

	\begin{minipage}[t]{9cm}
		\textbf{Amplitudenkontinuierlich}\\
		$x(t) = y$ mit $y \in \mathbb{R}$
		\begin{center}
			\includegraphics[width=4cm]{bilder/signal_amplitudenkontinuierlich.png}
       	\end{center}
	\end{minipage}
	&
	\begin{minipage}[t]{9cm}
		\textbf{Quantisiert}\\
		$x(t) = y_k$ mit $k \in \mathbb{K} \subset \mathbb{Z}$
		\begin{center}
			\includegraphics[width=4cm]{bilder/signal_quantisiert.png}
       	\end{center}
	\end{minipage}
\\
\hline

	\begin{minipage}[t]{9cm}
		\textbf{Analog} - \textit{zeit- und amplitudenkontinuierlich} \\

	\end{minipage}
	&
	\begin{minipage}[t]{9cm}
		\textbf{Digital} - \textit{zeitdiskret und quantisiert} \\

	\end{minipage}
\\
\hline
\end{tabular}
\renewcommand{\arraystretch}{\arraystretchOriginal}


	\subsection{Amplituden und Leistungen}
		\subsubsection{Normierte Signale}
			\begin{itemize}
				\item Absolute Werte des Signals sind irrelevant - Physikalische Dimension wird weggelassen.
				\item Signal $x(t)$ wird auf $|x_n(t)| \leq 1$ normiert.
			\end{itemize}
		\subsubsection{Signalbeschreibung}
			\begin{tabular}{p{4cm} p{7cm} p{7cm}}
				\hline
				\textbf{Linearer zeitlicher Mittelwert} & 
				$\bar{x}=\langle x(t)\rangle = \lim \limits_{T \to \infty} {\frac{1}{T} \int\limits_{-T/2}^{T/2} {x(t) dt}} $\\ \hline
				
				
				\textbf{Effektivwert}&
				$x_{RMS} = \sqrt{P_x} = \sqrt{\lim \limits_{T \to \infty} {\frac{1}{T} \int\limits_{-T/2}^{T/2} {|x(t)|^2 dt}}}$\\\hline
				
				\textbf{Crestfaktor / Scheitelfaktor} & 
				$C = \dfrac{|x|_{pk}}{x_{RMS}} = \dfrac{max(|x(t)|)}{\sqrt{Px}}$ &
				sinusförmige Signale $C = \sqrt{2}$\\\hline
				
				
				\textbf{Physikalische Leistung} &
				$P_{phys} = c \cdot P_x$\\
				& Bsp. $x(t) = i(t) \quad [A]$&
				$P_x = I_{RMS}^2 \quad [A^2]$ \qquad
				$P_{phys} = R \cdot I_{RMS}^2 \quad [W]$\\
				\hline
			\end{tabular}
			 
			

	\subsection{Pegel (Dezibel)}
		\textbf{Mit Dezibel (dB) vergleicht man Leistungspegel, nicht Amplituden!}
		
\begin{minipage}{14.5cm}
	
		\subsubsection{Relative Pegel}
		\begin{tabular}{p{3cm} p{5.5cm} p{2.5cm} p{3cm}}
			\textbf{Spannungspegel}&  
			$L_U = 20 \cdot \log_{10} \frac{U_x}{U_0}= 10 \cdot \log_{10} (\frac{U_x}{U_0})^2$ &
			$U_x = U_0 \cdot 10^{\frac{L}{20}} $ &
			$[L_U] = dB$\\
			
			\textbf{Strompegel}& $L_I  = 20 \cdot \log_{10} \frac{I_x}{I_0}$& $I_x = I_0 \cdot
			10^{\frac{L}{20}} $ & $[L_I] = dB$\\
			
			\textbf{Leistungspegel} & $L_P = 10 \cdot \log_{10} \frac{P_x}{P_0}$ & $P_x = P_0 \cdot
			10^{\frac{L}{10}} $ & $[L_P] = dB$ \\
			
			
			
	\end{tabular}

\end{minipage}
\scalebox{0.7}{\begin{minipage}[c]{4cm}
		%	\begin{right}
		\begin{tabular}{|r|r|r||r|r|r|}
			\hline
			dB & $\dfrac{P_{y}}{P_{x}}$ & $\dfrac{y_{rms}}{x_{rms}}$&dB & $\dfrac{P_{y}}{P_{x}}$ & $\dfrac{y_{rms}}{x_{rms}}$\\
			\hline
			20	&	100	&	10 &-20	& 	$\dfrac{1}{100}$	&	$\dfrac{1}{10}$	\\
			10	&	10	&$\sqrt{10}$& -10	& 	$\dfrac{1}{10}$	&	$\dfrac{1}{\sqrt{10}}$\\
			6	& 	4	&	2	&-6	& 	$\dfrac{1}{4}$	&	$\dfrac{1}{2}$\\
			3	& 	2	&	$\sqrt{2}$ & & &	\\
			0	& 	1	&	1 & -3	&	$\dfrac{1}{2}$	&	$\dfrac{1}{\sqrt{2}}$	\\
			
			\hline
		\end{tabular}
		%\end{center}
\end{minipage}}


		Beachte: Jeder relative Pegel gilt für U, I \& P. (Bei gleicher Referenz)
\subsubsection{Absolute Pegel}
	\begin{tabbing}
	xxxxxxxxxxxxxxxxxxxxxxxxx \= xxxxxxxxxxxxxxxxxxxxxxxxxxxxxxxxxxxxx  \= xxxxxxxxxxxxxxxxxxxxxxx \=\kill
	\textbf{dBW} \> Leistungspegel mit Bezugsgrösse $P_0 = $1W \> $\Longrightarrow \: 0 \: dBW = 1W = P_{0_{dBW}}$\\
	\textbf{dBm} \> Leistungspegel mit Bezugsgrösse $P_0 = $1mW \> $\Longrightarrow \: 0 \: dBm = 1mW = P_{0_{dBm}}$\\
	\textbf{dBV} \> Spannungspegel mit Bezugsgrösse $U_0 = $1V \> $\Longrightarrow \: 0 \: dBV = (1V)^2/R_{ref} = P_{0_{dBV}}$\\
	\textbf{dB}$\mu$V \> Spannungspegel mit Bezugsgrösse $U_0 = $1$\mu$V \> $\Longrightarrow \: 0 \: dB \mu V = (1 \mu V)^2/R_{ref} = P_{0_{dB\mu V}}$
	\end{tabbing}
Bei Leistungen in Bezug zu absoluten \textbf{Spannungspegeln} (dBV, dB$\mu$V) muss immer der \textbf{Referenz-Widerstand} ($R_{ref}$) berücksichtigt werden: \\
In der \textbf{HF-Technik} üblich, $R_{ref} = 50 \Omega \Rightarrow 0 dBV = 20 mW \qquad $
In der \textbf{Telefonie} üblich, $R_{ref} = 600  \Omega \Rightarrow 0 dBV =
1.67 mW$ 

\subsection{Fourier-Theorie}
	 \textbf{Siehe Zusammenfassung Integral Transformation}

\subsubsection{Ergänzung für NaT}
	\textbf{Symmetrie-Eigenschaft}\\
		Jedes Signal $x(t)$ kann aufgespaltet werden in:
		\parbox{6.5cm}{
			\begin{itemize}
				\item einen geraden (even) Anteil $x_e(t)$
				\item einen ungeraden (odd) Anteil $x_o(t)$
			\end{itemize}
		}
		$\qquad \Rightarrow x(t) = x_e(t) + x_o(t)$\\
		Es gilt:  $\quad$
		\parbox[t]{7cm}{
			$x_e(-t) = x_e(-t) : \qquad X_e(\omega)$ ist reellwertig\\
			\hspace{1cm} $x_o(-t) = -x_o(t) : \qquad X_o(\omega)$ ist imaginär
		}\\
		
	\begin{minipage}[t]{9cm}
		\textbf{Sinusf\"ormige Funktionen im Frequenzbereich}\\ 
			\hspace*{0.5cm}
			\parbox{8cm}{
				$\cos(\omega_0t) \; \laplace \; \pi \cdot (\delta(\omega - \omega_0)+ \delta(\omega + \omega_0))$\\
				$\sin(\omega_0t) \; \laplace \; -j\pi \cdot (\delta(\omega - \omega_0)- \delta(\omega + \omega_0))$	
			}
	\end{minipage}
	\begin{minipage}[t]{9cm}
		\textbf{Spektrum periodischer Leistungssignale}\\ 
			\hspace*{0.5cm}
			\parbox[b][1.2cm][c]{8cm}{
				$$\sum_{n = -\infty}^{+\infty} c_n \cdot e^{jn\omega_0t} \; \laplace \; 2\pi \sum_{n = -\infty}^{+\infty} c_n\delta(\omega - n\omega_0)$$		
			}
	\end{minipage}
