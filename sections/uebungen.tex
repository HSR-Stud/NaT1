\section{Übungsverzeichnis}

\subsection{Einführung}
	\begin{tabular}{|p{9cm}|p{2.5cm}|p{3.5cm}|p{2cm}|}
	\hline
	\textbf{Thema} & \textbf{Hausübung} & \textbf{Schaum} & \textbf{Praktikum} \\
	\hline
	Dezibel & 2.1 &  &  \\
	\hline
	Fouriertransformierte & 2.5, 2.6, 3.1, 3.2  & 1.10, 1.14, 1.15, 1.28 &  \\
	\hline
	Komplexe Fourierreihe & 2.3, 2.4 & & \\
	\hline
	Modulationssatz & 2.7 & & \\
	\hline
	Crest-Faktor & 1.1 & & \\
	\hline
	\end{tabular}

\subsection{LTI-Systeme}
	\begin{tabular}{|p{9cm}|p{2.5cm}|p{3.5cm}|p{2cm}|}
	\hline
	\textbf{Thema} & \textbf{Hausübung} & \textbf{Schaum} & \textbf{Praktikum} \\ \hline
	LTI: Linearität und Zeitinvarianz & 3.3, 3.4 & 2.1, 2.2 &  \\ \hline
	LTI: RL-Filter & 3.5 & 2.11 &  \\ \hline
	\end{tabular}

\subsection{Amplitudenmodulation}
	\begin{tabular}{|p{9cm}|p{2.5cm}|p{3.5cm}|p{2cm}|}
	\hline
	\textbf{Thema} & \textbf{Hausübung} & \textbf{Schaum} & \textbf{Praktikum} \\ \hline
	DSB-SC &  &  & 3-6.2 \\ \hline
	OAM: Koharände Demodulation & & 3.5 & 4 \\ \hline
	OAM: Spitzenwertdetektor & 5.2 & 3.6 & 4 \\ \hline
	Ordinary AM & 5.1 & 3.4 & 3 \\ \hline
	QAM & 6.3  & 3.15 & 4 \\ \hline
	SSB-AM: Demodulation mit koharäntem Demodulator & 5.1, 6.2 & 3.9 & 4 \\
	\hline SSB-AM: Modulation mit Eintonsignal & 6,1 & 3.7, 3.8 & 4-7.2 \\ \hline
	\end{tabular}

\subsection{Winkelmodulation - FM/PM}
	\begin{tabular}{|p{9cm}|p{2.5cm}|p{3.5cm}|p{2cm}|}
	\hline
	\textbf{Thema} & \textbf{Hausübung} & \textbf{Schaum} & \textbf{Praktikum} \\ \hline
	FM/PM: Bandbreite & 7.5, 7.6 & 4.9, 4.10 & 5-7.3 \\ \hline
	FM/PM: Bandbreitenänderung & 7.8 & 4.13 & 5-7.3 \\ \hline
	FM/PM: Frequenz- \& Phasenabweichung & 7.2 & 4.2 & 5-7.1 \\ \hline
	FM/PM: Leistung des modulierten Signals & 7.4 & 4.6 &  \\ \hline
	FM/PM: Modulationsindex & 7.6, 7.7 & 4.10, 4.11, 4.12 &  \\ \hline
	FM/PM: Momentanfrequenz & 7.1 & 4.1 & 5-7.1 \\ \hline
	PM: Nichtlinearität & 7.3 & 4.4 &  \\ \hline
	\end{tabular}

\subsection{Digitale Übermittlung analoger Signale}
	\begin{tabular}{|p{9cm}|p{2.5cm}|p{3.5cm}|p{2cm}|}
	\hline
	\textbf{Thema} & \textbf{Hausübung} & \textbf{Schaum} & \textbf{Praktikum} \\ \hline
	Delta-Modulation & 11 & 5.21, 5.22, 5.23, 5.24 & 7 \\ \hline
	ISI, Pulse-Shaping (Raised Cosine) &  & 5.36 &  \\ \hline
	PCM: Quantisierungsrauschen, SNR & 10.6 & 5.15, 5.16 &  \\ \hline
	PCM: Samplerate, Quantisierung, Bitrate & 10.3 & 5.12 &  \\ \hline
	PCM: Samplerate, Wortbreite, Bitdauer & 10.5 & 5.14 &  \\ \hline
	PCM: SNR und Bitrate einer CD & 10.8 & 5.17 &  \\ \hline
	PCM: Wortbreite & 10.4, 10.7 & 5.13, 5.16 &  \\ \hline
	Quantisierung: Ungleichförmig: A-Law, $\mu$-Law; SNR & 10.9, 10.10 & 5.18, 5.19
	& 6 \\ \hline Sampling: Basisbandsignale & 9.4a & 5.8 &  \\ \hline
	Sampling: Flat-top Sampling & 10.2 & 5.1 & 6 \\ \hline
	Sampling: Natural Sampling & 10.1 & 5.9 & 6 \\ \hline
	Sampling: Nyquist-Frequenz/-Rate/-Intervall & 9.1, 9.2 & 5.6 &  \\ \hline
	Sampling: Subsampling & 9.3, 9.4b & 5.7, 5.8 &  \\ \hline
	Sampling: Unterabtastung & & 5.3 &  \\ \hline
	Leistungscodierung & 12.1 && \\ \hline
	\end{tabular}


\subsection{Multiplexverfahren}
	\begin{tabular}{|p{9cm}|p{2.5cm}|p{3.5cm}|p{2cm}|}
	\hline
	\textbf{Thema} & \textbf{Hausübung} & \textbf{Schaum} & \textbf{Praktikum} \\ \hline
	Multiplexing: Zeitmultiplex (TDM) & 13 & 5.31 &  \\ \hline
	\end{tabular}
